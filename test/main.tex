\documentclass[preview,border=12pt,12pt]{article}
\usepackage{amsmath}
\usepackage{amssymb}
\usepackage{maxiplot}

\begin{document} %First we'll add some useful functions
\[\begin{maxima*}
indexpower(equ,var):= hipow(equ,var), /*for the highest power of a polynomial*/
powers(poly,var):= reverse(create_list(i,i,0,indexpower(poly,var))), /*to reverse the list of powers*/
polycoeff(poly,var):=map(lambda([i],coeff(poly,x,i)),powers(poly,var)), /*put all the coefficients in a list*/
systemeq(eq,var,list):=map("=",polycoeff(eq,var),list), /*to set the coefficients equal to a user entered list*/
printfunc(i,j):=print(concat(tex1(i),">0","&",tex1(j),"\\\\")), /*for outputting a list to a latex cases enviroment*/
load('to_poly_solve),
to_poly_solve(abs(x-17)=abs(2*x+8),x),
declare(c,posfun),
eq1:abs(x-2)+abs(x-10)=c,
cond1:to_poly_solve(eq1,[x,c]),
arglen:length(args(cond1)),
mylist:create_list(j,j,1,arglen),
rest(mylist,-1)
\end{maxima*}\]
Example 1.3.5. Given the equation $\imaxima{tex(abs(x-17)=abs(2*x+8))}$ we can determine the possible solutions as $\imaxima{mysol:to_poly_solve(abs(x-17)=abs(2*x+8),x),tex(setify(flatten(args(mysol))))}$\\
Example 1.3.6. Given the equation $\imaxima{tex(eq1)}$, determine any possible solutions.

This is really a case by case basis, which can be seen below,\\
$
\begin{maxima}
map(
    lambda([i],
        rest(args(
            args(cond1)[i]
        ),-1)
    ),
        mylist
    ),
varcond:map(flatten,
    map(
    lambda([j],
            map(
            lambda([k],
                flatten(args(k)
                )
            ),
            args(j[1])
        )
            ),
        map(
    lambda([i],
        rest(args(
            args(cond1)[i]
        ),-1)
    ),
        mylist
    )
)
),
sol1:map(
    lambda([j],args(j)[2]),
map(
    lambda([i],
        rest(args(
                args(cond1)[i]
            ),-1)
    ),
    mylist
)
),
map(lambda([i,j],print("\\begin{cases}"),map(printfunc,i,j),print("\\end{cases}\\\\")),varcond,sol1)
\end{maxima}$
\end{document}
