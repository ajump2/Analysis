\documentclass[12pt,a4paper]{article}
\usepackage[utf8]{inputenc}
\usepackage{amsmath}
\usepackage{amsthm}
\usepackage{amsfonts}
\usepackage{amssymb}
\usepackage{graphicx}
\usepackage{enumitem}
\usepackage[left=1cm,right=1cm,top=1cm,bottom=1cm]{geometry}
\setlength\parindent{0pt} %for no indentation
\usepackage{tikz}
\theoremstyle{plain}
\newtheorem{theorem}{Theorem}
\newtheorem{lemma}{Lemma}
\newtheorem{definition}{Definition}
\newtheorem{corollary}[theorem]{Corollary}
\newtheorem{case}{Case}
\newtheorem{subcase}{Subcase}[case]
\newtheorem{claim}{Claim}
\newtheorem{subclaim}{Subclaim}[claim]
\newtheorem{fact}{Fact}
\newtheorem{problem}{Problem}
\newtheorem{proposition}{Proposition}
\newtheorem{constr}{Construction}
\newtheorem{question}{Question}
\newtheorem{algo}{Algorithm}
\newtheorem{conjecture}{Conjecture}
\renewcommand{\qedsymbol}{\begin{flushright}
\sc q.e.d.
\end{flushright}}
\newtheorem{remark}{Remark}

\newcommand*\circled[1]{\tikz[baseline=(char.base)]{
   \node[shape=circle,draw,inner sep=1pt] (char) {#1};}}
\newcommand{\contra}{\[
\Rightarrow\!\Leftarrow
\]}
\newcommand*\abs[1]{
\left\vert #1 \right\vert
  }
\pagenumbering{gobble}
\begin{document}
Adam Jump\\
MATH 385\\
HW $\#1$\\

\begin{itemize}
\item Prove De Morgan's Law\\
  \proof{
$(A\cup B)^{\complement}=A^{\complement}\cap B^{\complement}$,\\
\begin{enumerate}
\item $(A\cup B)^{\complement}\subseteq A^{\complement}\cap B^{\complement}$
  Let $x\in(A\cup B)^{\complement}$,\\
  so $x\not\in(A\cup B)$, definition complement,\\
  and $x\not\in A$ or $x\not\in B$, definition $\cup$,\\
  \begin{enumerate}
  \item\label{A} $x\not\in A$\\
    then $x\in A^{\complement}$, definition complement,\\
    and $x\in A^{\complement}\cap B^{\complement}$ by composition
  \item Follows from \textbf{\ref{A}}.
  \end{enumerate}
  $\therefore (A\cup B)^{\complement}\subseteq A^{\complement}\cap B^{\complement}$
\item $A^{\complement}\cap B^{\complement}\subseteq (A\cup B)^{\complement}$\\
  Let $x\in A^{\complement}\cup B^{\complement}$,\\
  so $x\in A^{\complement} \text{ and } x\in B^{\complement}$ by definition $\cap$,\\
  which implies $x\not\in A$ or $x\not\in B$ by definition complement,\\
  and $x\not\in (A\cup B)$ definition $\cup$,\\
  so $x\in (A\cup B)^{\complement}$ definition complement,\\
  $\therefore (A^{\complement}\cap B^{\complement})\subseteq (A\cup B)^{\complement}$\\
  which means that $(A\cup B)^{\complement}=A^{\complement}\cap B^{\complement}$
\end{enumerate}
\qedsymbol
  }

\item Prove Triangle Inequality using cases\\
  Show $\abs{x+y} \leq \abs{x} + \abs{y}$

  \proof{$ $\newline
    This means that we have four cases to consider,
    \begin{enumerate}
    \item\label{case1} $x\geq 0, y\geq 0$
    \item\label{case2} $x\geq 0, y<0$
    \item\label{case3} $x<0, y\geq 0$
    \item\label{case4} $x<0, y<0$
    \end{enumerate}
    \textbf{\ref{case1}.} $\abs{x+y}=x+y$\\
    and\\
    $\abs{x}+\abs{y}=x+y$\\
    $\implies \abs{x+y}= \abs{x}+\abs{y}$\\
    and $\abs{x+y}\leq \abs{x}+\abs{y}$\\
    \textbf{\ref{case2}.}
    \begin{enumerate}
    \item $\abs{y}\geq x$\\
      $\abs{x+y}=-(x+y)$\\
      and\\
      $\abs{x}+\abs{y}=x-y$\\
      $\implies 0\leq x$\\
      and $0\leq 2x\implies 0\leq x+x$\\
      $\implies -y\leq x+x-y$\\
      $\implies -x-y\leq x-y$\\
      $\implies -(x+y)\leq x-y$\\
      $\implies \abs{x+y}=-(x+y)\leq x-y = \abs{x}+\abs{y}$\\
      $\implies \abs{x+y}\leq \abs{x}+\abs{y}$\\
      $\implies \abs{x+y}\leq \abs{x}+\abs{y}$\\
    \item $\abs{y}<x$
      which implies $\abs{x+y}=x+y$ and $\abs{x}+\abs{y}=x-y$,\\
      $y < 0$ by assumption,\\
      $\implies 2y < 0$,\\
      $x+y\leq x-y$,\\
    \end{enumerate}
    \textbf{\ref{case3}.} Follows from \ref{case2}\\
    \textbf{\ref{case4}.} $\abs{x+y} = -(x+y) = -x-y$\\
    and so $\abs{x}+\abs{y} = -x-y$\\
    $\implies \abs{x+y}=\abs{x}+\abs{y}$\\
    $\implies \abs{x+y}\leq\abs{x}+\abs{y}$\\
    \qedsymbol
    }
\item Example 1.3.5
  Determine the set $a=\left\{x~\vert~\abs{x-17}=\abs{2x+8}$
\item Example 1.3.6
  For $c\geq 0$, define the set $A_c=\{x~\vert~\abs{x-2}+\abs{x-10}=c\}$
\end{itemize}
\end{document}
